\documentclass[11pt, a4paper, fleqn]{article}
\usepackage{cp2425t}
\makeindex

%================= lhs2tex=====================================================%
%% ODER: format ==         = "\mathrel{==}"
%% ODER: format /=         = "\neq "
%
%
\makeatletter
\@ifundefined{lhs2tex.lhs2tex.sty.read}%
  {\@namedef{lhs2tex.lhs2tex.sty.read}{}%
   \newcommand\SkipToFmtEnd{}%
   \newcommand\EndFmtInput{}%
   \long\def\SkipToFmtEnd#1\EndFmtInput{}%
  }\SkipToFmtEnd

\newcommand\ReadOnlyOnce[1]{\@ifundefined{#1}{\@namedef{#1}{}}\SkipToFmtEnd}
\usepackage{amstext}
\usepackage{amssymb}
\usepackage{stmaryrd}
\DeclareFontFamily{OT1}{cmtex}{}
\DeclareFontShape{OT1}{cmtex}{m}{n}
  {<5><6><7><8>cmtex8
   <9>cmtex9
   <10><10.95><12><14.4><17.28><20.74><24.88>cmtex10}{}
\DeclareFontShape{OT1}{cmtex}{m}{it}
  {<-> ssub * cmtt/m/it}{}
\newcommand{\texfamily}{\fontfamily{cmtex}\selectfont}
\DeclareFontShape{OT1}{cmtt}{bx}{n}
  {<5><6><7><8>cmtt8
   <9>cmbtt9
   <10><10.95><12><14.4><17.28><20.74><24.88>cmbtt10}{}
\DeclareFontShape{OT1}{cmtex}{bx}{n}
  {<-> ssub * cmtt/bx/n}{}
\newcommand{\tex}[1]{\text{\texfamily#1}}	% NEU

\newcommand{\Sp}{\hskip.33334em\relax}


\newcommand{\Conid}[1]{\mathit{#1}}
\newcommand{\Varid}[1]{\mathit{#1}}
\newcommand{\anonymous}{\kern0.06em \vbox{\hrule\@width.5em}}
\newcommand{\plus}{\mathbin{+\!\!\!+}}
\newcommand{\bind}{\mathbin{>\!\!\!>\mkern-6.7mu=}}
\newcommand{\rbind}{\mathbin{=\mkern-6.7mu<\!\!\!<}}% suggested by Neil Mitchell
\newcommand{\sequ}{\mathbin{>\!\!\!>}}
\renewcommand{\leq}{\leqslant}
\renewcommand{\geq}{\geqslant}
\usepackage{polytable}

%mathindent has to be defined
\@ifundefined{mathindent}%
  {\newdimen\mathindent\mathindent\leftmargini}%
  {}%

\def\resethooks{%
  \global\let\SaveRestoreHook\empty
  \global\let\ColumnHook\empty}
\newcommand*{\savecolumns}[1][default]%
  {\g@addto@macro\SaveRestoreHook{\savecolumns[#1]}}
\newcommand*{\restorecolumns}[1][default]%
  {\g@addto@macro\SaveRestoreHook{\restorecolumns[#1]}}
\newcommand*{\aligncolumn}[2]%
  {\g@addto@macro\ColumnHook{\column{#1}{#2}}}

\resethooks

\newcommand{\onelinecommentchars}{\quad-{}- }
\newcommand{\commentbeginchars}{\enskip\{-}
\newcommand{\commentendchars}{-\}\enskip}

\newcommand{\visiblecomments}{%
  \let\onelinecomment=\onelinecommentchars
  \let\commentbegin=\commentbeginchars
  \let\commentend=\commentendchars}

\newcommand{\invisiblecomments}{%
  \let\onelinecomment=\empty
  \let\commentbegin=\empty
  \let\commentend=\empty}

\visiblecomments

\newlength{\blanklineskip}
\setlength{\blanklineskip}{0.66084ex}

\newcommand{\hsindent}[1]{\quad}% default is fixed indentation
\let\hspre\empty
\let\hspost\empty
\newcommand{\NB}{\textbf{NB}}
\newcommand{\Todo}[1]{$\langle$\textbf{To do:}~#1$\rangle$}

\EndFmtInput
\makeatother
%
%
%
%
%
%
% This package provides two environments suitable to take the place
% of hscode, called "plainhscode" and "arrayhscode". 
%
% The plain environment surrounds each code block by vertical space,
% and it uses \abovedisplayskip and \belowdisplayskip to get spacing
% similar to formulas. Note that if these dimensions are changed,
% the spacing around displayed math formulas changes as well.
% All code is indented using \leftskip.
%
% Changed 19.08.2004 to reflect changes in colorcode. Should work with
% CodeGroup.sty.
%
\ReadOnlyOnce{polycode.fmt}%
\makeatletter

\newcommand{\hsnewpar}[1]%
  {{\parskip=0pt\parindent=0pt\par\vskip #1\noindent}}

% can be used, for instance, to redefine the code size, by setting the
% command to \small or something alike
\newcommand{\hscodestyle}{}

% The command \sethscode can be used to switch the code formatting
% behaviour by mapping the hscode environment in the subst directive
% to a new LaTeX environment.

\newcommand{\sethscode}[1]%
  {\expandafter\let\expandafter\hscode\csname #1\endcsname
   \expandafter\let\expandafter\endhscode\csname end#1\endcsname}

% "compatibility" mode restores the non-polycode.fmt layout.

\newenvironment{compathscode}%
  {\par\noindent
   \advance\leftskip\mathindent
   \hscodestyle
   \let\\=\@normalcr
   \let\hspre\(\let\hspost\)%
   \pboxed}%
  {\endpboxed\)%
   \par\noindent
   \ignorespacesafterend}

\newcommand{\compaths}{\sethscode{compathscode}}

% "plain" mode is the proposed default.
% It should now work with \centering.
% This required some changes. The old version
% is still available for reference as oldplainhscode.

\newenvironment{plainhscode}%
  {\hsnewpar\abovedisplayskip
   \advance\leftskip\mathindent
   \hscodestyle
   \let\hspre\(\let\hspost\)%
   \pboxed}%
  {\endpboxed%
   \hsnewpar\belowdisplayskip
   \ignorespacesafterend}

\newenvironment{oldplainhscode}%
  {\hsnewpar\abovedisplayskip
   \advance\leftskip\mathindent
   \hscodestyle
   \let\\=\@normalcr
   \(\pboxed}%
  {\endpboxed\)%
   \hsnewpar\belowdisplayskip
   \ignorespacesafterend}

% Here, we make plainhscode the default environment.

\newcommand{\plainhs}{\sethscode{plainhscode}}
\newcommand{\oldplainhs}{\sethscode{oldplainhscode}}
\plainhs

% The arrayhscode is like plain, but makes use of polytable's
% parray environment which disallows page breaks in code blocks.

\newenvironment{arrayhscode}%
  {\hsnewpar\abovedisplayskip
   \advance\leftskip\mathindent
   \hscodestyle
   \let\\=\@normalcr
   \(\parray}%
  {\endparray\)%
   \hsnewpar\belowdisplayskip
   \ignorespacesafterend}

\newcommand{\arrayhs}{\sethscode{arrayhscode}}

% The mathhscode environment also makes use of polytable's parray 
% environment. It is supposed to be used only inside math mode 
% (I used it to typeset the type rules in my thesis).

\newenvironment{mathhscode}%
  {\parray}{\endparray}

\newcommand{\mathhs}{\sethscode{mathhscode}}

% texths is similar to mathhs, but works in text mode.

\newenvironment{texthscode}%
  {\(\parray}{\endparray\)}

\newcommand{\texths}{\sethscode{texthscode}}

% The framed environment places code in a framed box.

\def\codeframewidth{\arrayrulewidth}
\RequirePackage{calc}

\newenvironment{framedhscode}%
  {\parskip=\abovedisplayskip\par\noindent
   \hscodestyle
   \arrayrulewidth=\codeframewidth
   \tabular{@{}|p{\linewidth-2\arraycolsep-2\arrayrulewidth-2pt}|@{}}%
   \hline\framedhslinecorrect\\{-1.5ex}%
   \let\endoflinesave=\\
   \let\\=\@normalcr
   \(\pboxed}%
  {\endpboxed\)%
   \framedhslinecorrect\endoflinesave{.5ex}\hline
   \endtabular
   \parskip=\belowdisplayskip\par\noindent
   \ignorespacesafterend}

\newcommand{\framedhslinecorrect}[2]%
  {#1[#2]}

\newcommand{\framedhs}{\sethscode{framedhscode}}

% The inlinehscode environment is an experimental environment
% that can be used to typeset displayed code inline.

\newenvironment{inlinehscode}%
  {\(\def\column##1##2{}%
   \let\>\undefined\let\<\undefined\let\\\undefined
   \newcommand\>[1][]{}\newcommand\<[1][]{}\newcommand\\[1][]{}%
   \def\fromto##1##2##3{##3}%
   \def\nextline{}}{\) }%

\newcommand{\inlinehs}{\sethscode{inlinehscode}}

% The joincode environment is a separate environment that
% can be used to surround and thereby connect multiple code
% blocks.

\newenvironment{joincode}%
  {\let\orighscode=\hscode
   \let\origendhscode=\endhscode
   \def\endhscode{\def\hscode{\endgroup\def\@currenvir{hscode}\\}\begingroup}
   %\let\SaveRestoreHook=\empty
   %\let\ColumnHook=\empty
   %\let\resethooks=\empty
   \orighscode\def\hscode{\endgroup\def\@currenvir{hscode}}}%
  {\origendhscode
   \global\let\hscode=\orighscode
   \global\let\endhscode=\origendhscode}%

\makeatother
\EndFmtInput
%


%------------------------------------------------------------------------------%


%====== DEFINIR GRUPO E ELEMENTOS =============================================%

\group{G05}
\studentA{103993}{Júlia Bughi Corrêa da Costa }
\studentB{104171}{Gabriel Pereira Ribeiro }
\studentC{104613}{Luís Pinto da Cunha }

%==============================================================================%

\begin{document}

\sffamily
\setlength{\parindent}{0em}
\emergencystretch 3em
\renewcommand{\baselinestretch}{1.25} 
\input{Cover}
\pagestyle{pagestyle}
\setlength{\parindent}{1em}
\newgeometry{left=25mm,right=20mm,top=25mm,bottom=25mm}

\section*{Preâmbulo}

Em \CP\ pretende-se ensinar a progra\-mação de computadores como uma disciplina
científica. Para isso parte-se de um repertório de \emph{combinadores} que
formam uma álgebra da programação % (conjunto de leis universais e seus corolários)
e usam-se esses combinadores para construir programas \emph{composicionalmente},
isto é, agregando programas já existentes.

Na sequência pedagógica dos planos de estudo dos cursos que têm esta disciplina,
opta-se pela aplicação deste método à programação em \Haskell\ (sem prejuízo
da sua aplicação a outras linguagens funcionais). Assim, o presente trabalho
prático coloca os alunos perante problemas concretos que deverão ser implementados
em \Haskell. Há ainda um outro objectivo: o de ensinar a documentar programas,
a validá-los e a produzir textos técnico-científicos de qualidade.

Antes de abordarem os problemas propostos no trabalho, os grupos devem ler
com atenção o anexo \ref{sec:documentacao} onde encontrarão as instruções
relativas ao \emph{software} a instalar, etc.

Valoriza-se a escrita de \emph{pouco} código que corresponda a soluções simples
e elegantes que utilizem os combinadores de ordem superior estudados na disciplina.


\Problema

Esta questão aborda um problema que é conhecido pela designação '\emph{H-index of a Histogram}'
e que se formula facilmente:
\begin{quote}\em
O h-index de um histograma é o maior número \ensuremath{\Varid{n}} de barras do histograma cuja altura é maior ou igual a \ensuremath{\Varid{n}}.
\end{quote}
Por exemplo, o histograma 
\begin{hscode}\SaveRestoreHook
\column{B}{@{}>{\hspre}l<{\hspost}@{}}%
\column{E}{@{}>{\hspre}l<{\hspost}@{}}%
\>[B]{}\Varid{h}\mathrel{=}[\mskip1.5mu \mathrm{5},\mathrm{2},\mathrm{7},\mathrm{1},\mathrm{8},\mathrm{6},\mathrm{4},\mathrm{9}\mskip1.5mu]{}\<[E]%
\ColumnHook
\end{hscode}\resethooks
que se mostra na figura
	\histograma
tem \ensuremath{\Varid{hindex}\;\Varid{h}\mathrel{=}\mathrm{5}}
pois há \ensuremath{\mathrm{5}} colunas maiores que \ensuremath{\mathrm{5}}. (Não é \ensuremath{\mathrm{6}} pois maiores ou iguais que seis só há quatro.)

Pretende-se definida como um catamorfismo, anamorfismo ou hilomorfismo uma função em Haskell
\begin{hscode}\SaveRestoreHook
\column{B}{@{}>{\hspre}l<{\hspost}@{}}%
\column{E}{@{}>{\hspre}l<{\hspost}@{}}%
\>[B]{}\Varid{hindex}\mathbin{::}[\mskip1.5mu \Conid{Int}\mskip1.5mu]\to (\Conid{Int},[\mskip1.5mu \Conid{Int}\mskip1.5mu]){}\<[E]%
\ColumnHook
\end{hscode}\resethooks
tal que, para \ensuremath{(\Varid{i},\Varid{x})\mathrel{=}\Varid{hindex}\;\Varid{h}}, \ensuremath{\Varid{i}} é o H-index de \ensuremath{\Varid{h}} e \ensuremath{\Varid{x}} é a lista de colunas de \ensuremath{\Varid{h}} que para ele contribuem.

A proposta de \ensuremath{\Varid{hindex}} deverá vir acompanhada de um \textbf{diagrama} ilustrativo.

\Problema

Pelo \href{https://en.wikipedia.org/wiki/Fundamental_theorem_of_arithmetic}{teorema
fundamental da aritmética}, todo número inteiro positivo tem uma única factorização
prima.  For exemplo,
\begin{tabbing}\ttfamily
~primes~455\\
\ttfamily ~\char91{}5\char44{}7\char44{}13\char93{}\\
\ttfamily ~primes~433\\
\ttfamily ~\char91{}433\char93{}\\
\ttfamily ~primes~230\\
\ttfamily ~\char91{}2\char44{}5\char44{}23\char93{}
\end{tabbing}

\begin{enumerate}

\item	
Implemente como anamorfismo de listas a função
\begin{hscode}\SaveRestoreHook
\column{B}{@{}>{\hspre}l<{\hspost}@{}}%
\column{E}{@{}>{\hspre}l<{\hspost}@{}}%
\>[B]{}\Varid{primes}\mathbin{::}\mathbb{Z}\to [\mskip1.5mu \mathbb{Z}\mskip1.5mu]{}\<[E]%
\ColumnHook
\end{hscode}\resethooks
que deverá, recebendo um número inteiro positivo, devolver a respectiva lista
de factores primos.

A proposta de \ensuremath{\Varid{primes}} deverá vir acompanhada de um \textbf{diagrama} ilustrativo.

\item A figura mostra a ``\emph{árvore dos primos}'' dos números \ensuremath{[\mskip1.5mu \mathrm{455},\mathrm{669},\mathrm{6645},\mathrm{34},\mathrm{12},\mathrm{2}\mskip1.5mu]}.

      \primes

Com base na alínea anterior, implemente uma função em Haskell que faça a
geração de uma tal árvore a partir de uma lista de inteiros:

\begin{hscode}\SaveRestoreHook
\column{B}{@{}>{\hspre}l<{\hspost}@{}}%
\column{E}{@{}>{\hspre}l<{\hspost}@{}}%
\>[B]{}\Varid{prime\char95 tree}\mathbin{::}[\mskip1.5mu \mathbb{Z}\mskip1.5mu]\to \Conid{Exp}\;\mathbb{Z}\;\mathbb{Z}{}\<[E]%
\ColumnHook
\end{hscode}\resethooks

\textbf{Sugestão}: escreva o mínimo de código possível em \ensuremath{\Varid{prime\char95 tree}} investigando
cuidadosamente que funções disponíveis nas bibliotecas que são dadas podem
ser reutilizadas.%
\footnote{Pense sempre na sua produtividade quando está a programar --- essa
atitude será valorizada por qualquer empregador que vier a ter.}

\end{enumerate}

\Problema

A convolução \ensuremath{\Varid{a}\star \Varid{b}} de duas listas \ensuremath{\Varid{a}} e \ensuremath{\Varid{b}} --- uma operação relevante em computação
---  está muito bem explicada
\href{https://www.youtube.com/watch?v=KuXjwB4LzSA}{neste vídeo} do canal
\textbf{3Blue1Brown} do YouTube,
a partir de \href{https://www.youtube.com/watch?v=KuXjwB4LzSA&t=390s}{\ensuremath{\Varid{t}\mathrel{=}\mathrm{6}\mathbin{:}\mathrm{30}}}.
Aí se mostra como, por exemplo:
\begin{quote}
\ensuremath{[\mskip1.5mu \mathrm{1},\mathrm{2},\mathrm{3}\mskip1.5mu]\star [\mskip1.5mu \mathrm{4},\mathrm{5},\mathrm{6}\mskip1.5mu]\mathrel{=}[\mskip1.5mu \mathrm{4},\mathrm{13},\mathrm{28},\mathrm{27},\mathrm{18}\mskip1.5mu]} 
\end{quote}
A solução abaixo, proposta pelo chatGPT,
\begin{hscode}\SaveRestoreHook
\column{B}{@{}>{\hspre}l<{\hspost}@{}}%
\column{3}{@{}>{\hspre}l<{\hspost}@{}}%
\column{E}{@{}>{\hspre}l<{\hspost}@{}}%
\>[B]{}\Varid{convolve}\mathbin{::}\Conid{Num}\;\Varid{a}\Rightarrow [\mskip1.5mu \Varid{a}\mskip1.5mu]\to [\mskip1.5mu \Varid{a}\mskip1.5mu]\to [\mskip1.5mu \Varid{a}\mskip1.5mu]{}\<[E]%
\\
\>[B]{}\Varid{convolve}\;\Varid{xs}\;\Varid{ys}\mathrel{=}[\mskip1.5mu \Varid{sum}\mathbin{\$}\Varid{zipWith}\;(\mathbin{*})\;(\Varid{take}\;\Varid{n}\;(\Varid{drop}\;\Varid{i}\;\Varid{xs}))\;\Varid{ys}\mid \Varid{i}\leftarrow [\mskip1.5mu \mathrm{0}\mathinner{\ldotp\ldotp}(\length \;\Varid{xs}\mathbin{-}\Varid{n})\mskip1.5mu]\mskip1.5mu]{}\<[E]%
\\
\>[B]{}\hsindent{3}{}\<[3]%
\>[3]{}\mathbf{where}\;\Varid{n}\mathrel{=}\length \;\Varid{ys}{}\<[E]%
\ColumnHook
\end{hscode}\resethooks
está manifestamente errada, pois \ensuremath{\Varid{convolve}\;[\mskip1.5mu \mathrm{1},\mathrm{2},\mathrm{3}\mskip1.5mu]\;[\mskip1.5mu \mathrm{4},\mathrm{5},\mathrm{6}\mskip1.5mu]\mathrel{=}[\mskip1.5mu \mathrm{32}\mskip1.5mu]} (!).

Proponha, explicando-a devidamente, uma solução sua para \ensuremath{\Varid{convolve}}.
Valorizar-se-á a economia de código e o recurso aos combinadores \emph{pointfree} estudados
na disciplina, em particular a triologia \emph{ana-cata-hilo} de tipos disponíveis nas
bibliotecas dadas ou a definir.

\Problema

Considere-se a seguinte sintaxe (abstrata e simplificada) para \textbf{expressões numéricas} (em \ensuremath{\Varid{b}}) com variáveis (em \ensuremath{\Varid{a}}),
\begin{hscode}\SaveRestoreHook
\column{B}{@{}>{\hspre}l<{\hspost}@{}}%
\column{19}{@{}>{\hspre}l<{\hspost}@{}}%
\column{50}{@{}>{\hspre}l<{\hspost}@{}}%
\column{E}{@{}>{\hspre}l<{\hspost}@{}}%
\>[B]{}\mathbf{data}\;\Conid{Expr}\;\Varid{b}\;\Varid{a}\mathrel{=}{}\<[19]%
\>[19]{}\Conid{V}\;\Varid{a}\mid \Conid{N}\;\Varid{b}\mid \Conid{T}\;\Conid{Op}\;[\mskip1.5mu \Conid{Expr}\;\Varid{b}\;\Varid{a}\mskip1.5mu]\;{}\<[50]%
\>[50]{}\mathbf{deriving}\;(\Conid{Show},\Conid{Eq}){}\<[E]%
\\[\blanklineskip]%
\>[B]{}\mathbf{data}\;\Conid{Op}\mathrel{=}\Conid{ITE}\mid \Conid{Add}\mid \Conid{Mul}\mid \Conid{Suc}\;\mathbf{deriving}\;(\Conid{Show},\Conid{Eq}){}\<[E]%
\ColumnHook
\end{hscode}\resethooks
possivelmente condicionais (cf.\ \ensuremath{\Conid{ITE}}, i.e.\ o operador condicional ``if-then-else``).
Por exemplo, a árvore mostrada a seguir
        \treeA
representa a expressão
\begin{eqnarray}
        \ensuremath{\Varid{ite}\;(\Conid{V}\;\text{\ttfamily \char34 x\char34})\;(\Conid{N}\;\mathrm{0})\;(\Varid{multi}\;(\Conid{V}\;\text{\ttfamily \char34 y\char34})\;(\Varid{soma}\;(\Conid{N}\;\mathrm{3})\;(\Conid{V}\;\text{\ttfamily \char34 y\char34})))}
        \label{eq:expr}
\end{eqnarray}
--- i.e.\ \ensuremath{\mathbf{if}\;\Varid{x}\;\mathbf{then}\;\mathrm{0}\;\mathbf{else}\;\Varid{y}\mathbin{*}(\mathrm{3}\mathbin{+}\Varid{y})} ---
assumindo as ``helper functions'':
\begin{hscode}\SaveRestoreHook
\column{B}{@{}>{\hspre}l<{\hspost}@{}}%
\column{7}{@{}>{\hspre}l<{\hspost}@{}}%
\column{E}{@{}>{\hspre}l<{\hspost}@{}}%
\>[B]{}\Varid{soma}\;{}\<[7]%
\>[7]{}\Varid{x}\;\Varid{y}\mathrel{=}\Conid{T}\;\Conid{Add}\;[\mskip1.5mu \Varid{x},\Varid{y}\mskip1.5mu]{}\<[E]%
\\
\>[B]{}\Varid{multi}\;\Varid{x}\;\Varid{y}\mathrel{=}\Conid{T}\;\Conid{Mul}\;[\mskip1.5mu \Varid{x},\Varid{y}\mskip1.5mu]{}\<[E]%
\\
\>[B]{}\Varid{ite}\;\Varid{x}\;\Varid{y}\;\Varid{z}\mathrel{=}\Conid{T}\;\Conid{ITE}\;[\mskip1.5mu \Varid{x},\Varid{y},\Varid{z}\mskip1.5mu]{}\<[E]%
\ColumnHook
\end{hscode}\resethooks

No anexo \ref{sec:codigo} propôe-se uma base para o tipo Expr (\ensuremath{\Varid{baseExpr}}) e a 
correspondente algebra \ensuremath{\Varid{inExpr}} para construção do tipo \ensuremath{\Conid{Expr}}.

\begin{enumerate}
\item        Complete as restantes definições da biblioteca \ensuremath{\Conid{Expr}}  pedidas no anexo \ref{sec:resolucao}.
\item        No mesmo anexo, declare \ensuremath{\Conid{Expr}\;\Varid{b}} como instância da classe \ensuremath{\Conid{Monad}}. \textbf{Sugestão}: relembre os exercícios da ficha 12.
\item        Defina como um catamorfismo de \ensuremath{\Conid{Expr}} a sua versão monádia, que deverá ter o tipo:
\begin{hscode}\SaveRestoreHook
\column{B}{@{}>{\hspre}l<{\hspost}@{}}%
\column{E}{@{}>{\hspre}l<{\hspost}@{}}%
\>[B]{}\Varid{mcataExpr}\mathbin{::}\Conid{Monad}\;\Varid{m}\Rightarrow (\Varid{a}+(\Varid{b}+(\Conid{Op},\Varid{m}\;[\mskip1.5mu \Varid{c}\mskip1.5mu]))\to \Varid{m}\;\Varid{c})\to \Conid{Expr}\;\Varid{b}\;\Varid{a}\to \Varid{m}\;\Varid{c}{}\<[E]%
\ColumnHook
\end{hscode}\resethooks
\item        Para se avaliar uma expressão é preciso que todas as suas variáveis estejam instanciadas.
Complete a definição da função
\begin{hscode}\SaveRestoreHook
\column{B}{@{}>{\hspre}l<{\hspost}@{}}%
\column{E}{@{}>{\hspre}l<{\hspost}@{}}%
\>[B]{}\Varid{let\char95 exp}\mathbin{::}(\Conid{Num}\;\Varid{c})\Rightarrow (\Varid{a}\to \Conid{Expr}\;\Varid{c}\;\Varid{b})\to \Conid{Expr}\;\Varid{c}\;\Varid{a}\to \Conid{Expr}\;\Varid{c}\;\Varid{b}{}\<[E]%
\ColumnHook
\end{hscode}\resethooks
que, dada uma expressão com variáveis em \ensuremath{\Varid{a}} e uma função que a cada uma dessas variáveis atribui uma
expressão (\ensuremath{\Varid{a}\to \Conid{Expr}\;\Varid{c}\;\Varid{b}}), faz a correspondente substituição.\footnote{Cf.\ expressões \ensuremath{\mathbf{let}\mathbin{...}\mathbf{in}\mathbin{...}}.}
Por exemplo, dada
\begin{hscode}\SaveRestoreHook
\column{B}{@{}>{\hspre}l<{\hspost}@{}}%
\column{7}{@{}>{\hspre}l<{\hspost}@{}}%
\column{E}{@{}>{\hspre}l<{\hspost}@{}}%
\>[B]{}\Varid{f}\;\text{\ttfamily \char34 x\char34}\mathrel{=}\Conid{N}\;\mathrm{0}{}\<[E]%
\\
\>[B]{}\Varid{f}\;\text{\ttfamily \char34 y\char34}\mathrel{=}\Conid{N}\;\mathrm{5}{}\<[E]%
\\
\>[B]{}\Varid{f}\;\anonymous {}\<[7]%
\>[7]{}\mathrel{=}\Conid{N}\;\mathrm{99}{}\<[E]%
\ColumnHook
\end{hscode}\resethooks
ter-se-á
\begin{hscode}\SaveRestoreHook
\column{B}{@{}>{\hspre}l<{\hspost}@{}}%
\column{9}{@{}>{\hspre}l<{\hspost}@{}}%
\column{E}{@{}>{\hspre}l<{\hspost}@{}}%
\>[9]{}\Varid{let\char95 exp}\;\Varid{f}\;\Varid{e}\mathrel{=}\Conid{T}\;\Conid{ITE}\;[\mskip1.5mu \Conid{N}\;\mathrm{1},\Conid{N}\;\mathrm{0},\Conid{T}\;\Conid{Mul}\;[\mskip1.5mu \Conid{N}\;\mathrm{5},\Conid{T}\;\Conid{Add}\;[\mskip1.5mu \Conid{N}\;\mathrm{3},\Conid{N}\;\mathrm{1}\mskip1.5mu]\mskip1.5mu]\mskip1.5mu]{}\<[E]%
\ColumnHook
\end{hscode}\resethooks
isto é, a árvore da figura a seguir: 
        \treeB

\item Finalmente, defina a função de avaliação de uma expressão, com tipo

\begin{hscode}\SaveRestoreHook
\column{B}{@{}>{\hspre}l<{\hspost}@{}}%
\column{32}{@{}>{\hspre}l<{\hspost}@{}}%
\column{E}{@{}>{\hspre}l<{\hspost}@{}}%
\>[B]{}\Varid{evaluate}\mathbin{::}(\Conid{Num}\;\Varid{a},\Conid{Ord}\;\Varid{a})\Rightarrow {}\<[32]%
\>[32]{}\Conid{Expr}\;\Varid{a}\;\Varid{b}\to \Conid{Maybe}\;\Varid{a}{}\<[E]%
\ColumnHook
\end{hscode}\resethooks

que deverá ter em conta as seguintes situações de erro:

\begin{enumerate}

\item \emph{Variáveis} --- para ser avaliada, \ensuremath{\Varid{x}} em \ensuremath{\Varid{evaluate}\;\Varid{x}} não pode conter variáveis. Assim, por exemplo,
        \begin{hscode}\SaveRestoreHook
\column{B}{@{}>{\hspre}l<{\hspost}@{}}%
\column{9}{@{}>{\hspre}l<{\hspost}@{}}%
\column{E}{@{}>{\hspre}l<{\hspost}@{}}%
\>[9]{}\Varid{evaluate}\;\Varid{e}\mathrel{=}\Conid{Nothing}{}\<[E]%
\\
\>[9]{}\Varid{evaluate}\;(\Varid{let\char95 exp}\;\Varid{f}\;\Varid{e})\mathrel{=}\Conid{Just}\;\mathrm{40}{}\<[E]%
\ColumnHook
\end{hscode}\resethooks
para \ensuremath{\Varid{f}} e \ensuremath{\Varid{e}}  dadas acima.

\item \emph{Aridades} --- todas as ocorrências dos operadores deverão ter
      o devido número de sub-expressões, por exemplo:
        \begin{hscode}\SaveRestoreHook
\column{B}{@{}>{\hspre}l<{\hspost}@{}}%
\column{9}{@{}>{\hspre}l<{\hspost}@{}}%
\column{E}{@{}>{\hspre}l<{\hspost}@{}}%
\>[9]{}\Varid{evaluate}\;(\Conid{T}\;\Conid{Add}\;[\mskip1.5mu \Conid{N}\;\mathrm{2},\Conid{N}\;\mathrm{3}\mskip1.5mu])\mathrel{=}\Conid{Just}\;\mathrm{5}{}\<[E]%
\\
\>[9]{}\Varid{evaluate}\;(\Conid{T}\;\Conid{Mul}\;[\mskip1.5mu \Conid{N}\;\mathrm{2}\mskip1.5mu])\mathrel{=}\Conid{Nothing}{}\<[E]%
\ColumnHook
\end{hscode}\resethooks

\end{enumerate}

\end{enumerate}

\noindent
\textbf{Sugestão}: de novo se insiste na escrita do mínimo de código possível,
tirando partido da riqueza estrutural do tipo \ensuremath{\Conid{Expr}} que é assunto desta questão.
Sugere-se também o recurso a diagramas para explicar as soluções propostas.

\part*{Anexos}

\appendix

\section{Natureza do trabalho a realizar}
\label{sec:documentacao}
Este trabalho teórico-prático deve ser realizado por grupos de 3 alunos.
Os detalhes da avaliação (datas para submissão do relatório e sua defesa
oral) são os que forem publicados na \cp{página da disciplina} na \emph{internet}.

Recomenda-se uma abordagem participativa dos membros do grupo em \textbf{todos}
os exercícios do trabalho, para assim poderem responder a qualquer questão
colocada na \emph{defesa oral} do relatório.

Para cumprir de forma integrada os objectivos do trabalho vamos recorrer
a uma técnica de programa\-ção dita ``\litp{literária}'' \cite{Kn92}, cujo
princípio base é o seguinte:
%
\begin{quote}\em
	Um programa e a sua documentação devem coincidir.
\end{quote}
%
Por outras palavras, o \textbf{código fonte} e a \textbf{documentação} de um
programa deverão estar no mesmo ficheiro.

O ficheiro \texttt{cp2425t.pdf} que está a ler é já um exemplo de
\litp{programação literária}: foi gerado a partir do texto fonte
\texttt{cp2425t.lhs}\footnote{O sufixo `lhs' quer dizer
\emph{\lhaskell{literate Haskell}}.} que encontrará no \MaterialPedagogico\
desta disciplina des\-com\-pactando o ficheiro \texttt{cp2425t.zip}.

Como se mostra no esquema abaixo, de um único ficheiro (\ensuremath{\Varid{lhs}})
gera-se um PDF ou faz-se a interpretação do código \Haskell\ que ele inclui:

	\esquema

Vê-se assim que, para além do \GHCi, serão necessários os executáveis \PdfLatex\ e
\LhsToTeX. Para facilitar a instalação e evitar problemas de versões e
conflitos com sistemas operativos, é recomendado o uso do \Docker\ tal como
a seguir se descreve.

\section{Docker} \label{sec:docker}

Recomenda-se o uso do \container\ cuja imagem é gerada pelo \Docker\ a partir do ficheiro
\texttt{Dockerfile} que se encontra na diretoria que resulta de descompactar
\texttt{cp2425t.zip}. Este \container\ deverá ser usado na execução
do \GHCi\ e dos comandos relativos ao \Latex. (Ver também a \texttt{Makefile}
que é disponibilizada.)

Após \href{https://docs.docker.com/engine/install/}{instalar o Docker} e
descarregar o referido zip com o código fonte do trabalho,
basta executar os seguintes comandos:
\begin{Verbatim}[fontsize=\small]
    $ docker build -t cp2425t .
    $ docker run -v ${PWD}:/cp2425t -it cp2425t
\end{Verbatim}
\textbf{NB}: O objetivo é que o container\ seja usado \emph{apenas} 
para executar o \GHCi\ e os comandos relativos ao \Latex.
Deste modo, é criado um \textit{volume} (cf.\ a opção \texttt{-v \$\{PWD\}:/cp2425t}) 
que permite que a diretoria em que se encontra na sua máquina local 
e a diretoria \texttt{/cp2425t} no \container\ sejam partilhadas.

Pretende-se então que visualize/edite os ficheiros na sua máquina local e que
os compile no \container, executando:
\begin{Verbatim}[fontsize=\small]
    $ lhs2TeX cp2425t.lhs > cp2425t.tex
    $ pdflatex cp2425t
\end{Verbatim}
\LhsToTeX\ é o pre-processador que faz ``pretty printing'' de código Haskell
em \Latex\ e que faz parte já do \container. Alternativamente, basta executar
\begin{Verbatim}[fontsize=\small]
    $ make
\end{Verbatim}
para obter o mesmo efeito que acima.

Por outro lado, o mesmo ficheiro \texttt{cp2425t.lhs} é executável e contém
o ``kit'' básico, escrito em \Haskell, para realizar o trabalho. Basta executar
\begin{Verbatim}[fontsize=\small]
    $ ghci cp2425t.lhs
\end{Verbatim}

\noindent Abra o ficheiro \texttt{cp2425t.lhs} no seu editor de texto preferido
e verifique que assim é: todo o texto que se encontra dentro do ambiente
\begin{quote}\small\tt
\text{\ttfamily \char92{}begin\char123{}code\char125{}}
\\ ... \\
\text{\ttfamily \char92{}end\char123{}code\char125{}}
\end{quote}
é seleccionado pelo \GHCi\ para ser executado.

\section{Em que consiste o TP}

Em que consiste, então, o \emph{relatório} a que se referiu acima?
É a edição do texto que está a ser lido, preenchendo o anexo \ref{sec:resolucao}
com as respostas. O relatório deverá conter ainda a identificação dos membros
do grupo de trabalho, no local respectivo da folha de rosto.

Para gerar o PDF integral do relatório deve-se ainda correr os comando seguintes,
que actualizam a bibliografia (com \Bibtex) e o índice remissivo (com \Makeindex),
\begin{Verbatim}[fontsize=\small]
    $ bibtex cp2425t.aux
    $ makeindex cp2425t.idx
\end{Verbatim}
e recompilar o texto como acima se indicou. (Como já se disse, pode fazê-lo
correndo simplesmente \texttt{make} no \container.)

No anexo \ref{sec:codigo} disponibiliza-se algum código \Haskell\ relativo
aos problemas que são colocados. Esse anexo deverá ser consultado e analisado
à medida que isso for necessário.

Deve ser feito uso da \litp{programação literária} para documentar bem o código que se
desenvolver, em particular fazendo diagramas explicativos do que foi feito e
tal como se explica no anexo \ref{sec:diagramas} que se segue.

\section{Como exprimir cálculos e diagramas em LaTeX/lhs2TeX} \label{sec:diagramas}

Como primeiro exemplo, estudar o texto fonte (\lhstotex{lhs}) do que está a ler\footnote{
Procure e.g.\ por \texttt{"sec:diagramas"}.} onde se obtém o efeito seguinte:\footnote{Exemplos
tirados de \cite{Ol18}.}
\begin{eqnarray*}
\start
\ensuremath{\Varid{id}\mathrel{=}\conj{\Varid{f}}{\Varid{g}}}
\just\equiv{ universal property }
\ensuremath{\begin{lcbr}\p1\comp \Varid{id}\mathrel{=}\Varid{f}\\\p2\comp \Varid{id}\mathrel{=}\Varid{g}\end{lcbr}}
\just\equiv{ identity }
\ensuremath{\begin{lcbr}\p1\mathrel{=}\Varid{f}\\\p2\mathrel{=}\Varid{g}\end{lcbr}}
\qed
\end{eqnarray*}

Os diagramas podem ser produzidos recorrendo à \emph{package} \Xymatrix, por exemplo:
\begin{eqnarray*}
\xymatrix@C=2cm{
    \ensuremath{\N_0}
           \ar[d]_-{\ensuremath{\cataNat{\Varid{g}}}}
&
    \ensuremath{\mathrm{1}\mathbin{+}\N_0}
           \ar[d]^{\ensuremath{\Varid{id}\mathbin{+}\cataNat{\Varid{g}}}}
           \ar[l]_-{\ensuremath{\mathsf{in}}}
\\
     \ensuremath{\Conid{B}}
&
     \ensuremath{\mathrm{1}\mathbin{+}\Conid{B}}
           \ar[l]^-{\ensuremath{\Varid{g}}}
}
\end{eqnarray*}

\section{Código fornecido}\label{sec:codigo}

\subsection*{Problema 1}

\begin{hscode}\SaveRestoreHook
\column{B}{@{}>{\hspre}l<{\hspost}@{}}%
\column{E}{@{}>{\hspre}l<{\hspost}@{}}%
\>[B]{}\Varid{h}\mathbin{::}[\mskip1.5mu \Conid{Int}\mskip1.5mu]{}\<[E]%
\ColumnHook
\end{hscode}\resethooks

\subsection*{Problema 4}
Definição do tipo:
\begin{hscode}\SaveRestoreHook
\column{B}{@{}>{\hspre}l<{\hspost}@{}}%
\column{E}{@{}>{\hspre}l<{\hspost}@{}}%
\>[B]{}\Varid{inExpr}\mathrel{=}\alt{\Conid{V}}{\alt{\Conid{N}}{\uncurry{\Conid{T}}}}{}\<[E]%
\\[\blanklineskip]%
\>[B]{}\Varid{baseExpr}\;\Varid{g}\;\Varid{h}\;\Varid{f}\mathrel{=}\Varid{g}+(\Varid{h}+\Varid{id}\times\map \;\Varid{f}){}\<[E]%
\ColumnHook
\end{hscode}\resethooks
Exemplos de expressões:
\begin{hscode}\SaveRestoreHook
\column{B}{@{}>{\hspre}l<{\hspost}@{}}%
\column{E}{@{}>{\hspre}l<{\hspost}@{}}%
\>[B]{}\Varid{e}\mathrel{=}\Varid{ite}\;(\Conid{V}\;\text{\ttfamily \char34 x\char34})\;(\Conid{N}\;\mathrm{0})\;(\Varid{multi}\;(\Conid{V}\;\text{\ttfamily \char34 y\char34})\;(\Varid{soma}\;(\Conid{N}\;\mathrm{3})\;(\Conid{V}\;\text{\ttfamily \char34 y\char34}))){}\<[E]%
\\
\>[B]{}\Varid{i}\mathrel{=}\Varid{ite}\;(\Conid{V}\;\text{\ttfamily \char34 x\char34})\;(\Conid{N}\;\mathrm{1})\;(\Varid{multi}\;(\Conid{V}\;\text{\ttfamily \char34 y\char34})\;(\Varid{soma}\;(\Conid{N}\;(\mathrm{3}\mathbin{/}\mathrm{5}))\;(\Conid{V}\;\text{\ttfamily \char34 y\char34}))){}\<[E]%
\ColumnHook
\end{hscode}\resethooks
Exemplo de teste:
\begin{hscode}\SaveRestoreHook
\column{B}{@{}>{\hspre}l<{\hspost}@{}}%
\column{5}{@{}>{\hspre}l<{\hspost}@{}}%
\column{E}{@{}>{\hspre}l<{\hspost}@{}}%
\>[B]{}\Varid{teste}\mathrel{=}\Varid{evaluate}\;(\Varid{let\char95 exp}\;\Varid{f}\;\Varid{i})\equiv \Conid{Just}\;(\mathrm{26}\mathbin{/}\mathrm{245}){}\<[E]%
\\
\>[B]{}\hsindent{5}{}\<[5]%
\>[5]{}\mathbf{where}\;\Varid{f}\;\text{\ttfamily \char34 x\char34}\mathrel{=}\Conid{N}\;\mathrm{0};\Varid{f}\;\text{\ttfamily \char34 y\char34}\mathrel{=}\Conid{N}\;(\mathrm{1}\mathbin{/}\mathrm{7}){}\<[E]%
\ColumnHook
\end{hscode}\resethooks

%----------------- Soluções dos alunos -----------------------------------------%

\section{Soluções dos alunos}\label{sec:resolucao}
Os alunos devem colocar neste anexo as suas soluções para os exercícios
propostos, de acordo com o ``layout'' que se fornece.
Não podem ser alterados os nomes ou tipos das funções dadas, mas pode ser
adicionado texto ao anexo, bem como diagramas e/ou outras funções auxiliares
que sejam necessárias.

\noindent
\textbf{Importante}: Não pode ser alterado o texto deste ficheiro fora deste anexo.

\subsection*{Problema 1}
Para a resolução deste problema, utilizamos uma abordagem em duas fases, \ensuremath{\Varid{divide}} e \ensuremath{\Varid{conquer}}, típica dos hilomorfismos.

\begin{hscode}\SaveRestoreHook
\column{B}{@{}>{\hspre}l<{\hspost}@{}}%
\column{19}{@{}>{\hspre}l<{\hspost}@{}}%
\column{E}{@{}>{\hspre}l<{\hspost}@{}}%
\>[B]{}\Varid{divide}\mathbin{::}[\mskip1.5mu \Conid{Int}\mskip1.5mu]\to ()+((\Conid{Int},[\mskip1.5mu \Conid{Int}\mskip1.5mu]),[\mskip1.5mu \Conid{Int}\mskip1.5mu]){}\<[E]%
\\
\>[B]{}\Varid{divide}\mathrel{=}(\Varid{id}+{}\<[19]%
\>[19]{}(\conj{\conj{\Varid{head}}{\Varid{id}}}{\Varid{tail}}\comp \Varid{cons}))\comp \Varid{outList}\comp \Varid{iSort}{}\<[E]%
\ColumnHook
\end{hscode}\resethooks
\begin{hscode}\SaveRestoreHook
\column{B}{@{}>{\hspre}l<{\hspost}@{}}%
\column{E}{@{}>{\hspre}l<{\hspost}@{}}%
\>[B]{}\Varid{pp1}\mathrel{=}\uncurry{(\not\equiv )}\comp \conj{\p1\comp \p2}{\underline{\mathrm{0}}}{}\<[E]%
\\
\>[B]{}\Varid{pp2}\mathrel{=}\uncurry{(\leq )}\comp \conj{\p1\comp \p2}{\length \comp \p2\comp \p2}{}\<[E]%
\ColumnHook
\end{hscode}\resethooks
\begin{hscode}\SaveRestoreHook
\column{B}{@{}>{\hspre}l<{\hspost}@{}}%
\column{E}{@{}>{\hspre}l<{\hspost}@{}}%
\>[B]{}\Varid{conquer}\mathbin{::}()+((\Conid{Int},[\mskip1.5mu \Conid{Int}\mskip1.5mu]),(\Conid{Int},[\mskip1.5mu \Conid{Int}\mskip1.5mu]))\to (\Conid{Int},[\mskip1.5mu \Conid{Int}\mskip1.5mu]){}\<[E]%
\\
\>[B]{}\Varid{conquer}\mathrel{=}\alt{\conj{\underline{\mathrm{0}}}{\Varid{nil}}}{\Varid{cond}\;(\uncurry{(\mathrel{\wedge})}\comp \conj{\Varid{pp1}}{\Varid{pp2}})\;\p2\;\p1}{}\<[E]%
\ColumnHook
\end{hscode}\resethooks
\begin{hscode}\SaveRestoreHook
\column{B}{@{}>{\hspre}l<{\hspost}@{}}%
\column{E}{@{}>{\hspre}l<{\hspost}@{}}%
\>[B]{}\Varid{hindex}\mathrel{=}\Varid{hyloList}\;\Varid{conquer}\;\Varid{divide}{}\<[E]%
\ColumnHook
\end{hscode}\resethooks

\begin{eqnarray*}
\xymatrix@C=3.1cm@R=2cm{
    \ensuremath{[\mskip1.5mu \mathbb{Z}\mskip1.5mu]}
           \ar[d]_-{\ensuremath{\ana{\Varid{divide}}}}
           \ar[r]^-{\ensuremath{\Varid{iSort}\comp \Varid{outList}}}
&
    \ensuremath{\mathrm{1}\mathbin{+}\mathbb{Z}\times[\mskip1.5mu \mathbb{Z}\mskip1.5mu]}
            \ar[r]^-{\ensuremath{\Varid{id}\mathbin{+}(\conj{\conj{\Varid{head}}{\Varid{id}}}{\Varid{tail}}\comp \Varid{cons})}}
&
    \ensuremath{\mathrm{1}\mathbin{+}(\mathbb{Z}\times[\mskip1.5mu \mathbb{Z}\mskip1.5mu])\times[\mskip1.5mu \mathbb{Z}\mskip1.5mu]}
              \ar[d]^-{\ensuremath{\Varid{id}\mathbin{+}(\Varid{id}\times\Varid{id})\times\ana{\Varid{divide}}}}
\\
     \ensuremath{[\mskip1.5mu \mathbb{Z}\times[\mskip1.5mu \mathbb{Z}\mskip1.5mu]\mskip1.5mu]}
            \ar[d]_-{\ensuremath{\llparenthesis\, \Varid{conquer}\,\rrparenthesis}}
            \ar[rr]^-{\ensuremath{\Varid{out}}}
&
&
     \ensuremath{\mathrm{1}\mathbin{+}(\mathbb{Z}\times[\mskip1.5mu \mathbb{Z}\mskip1.5mu])\times[\mskip1.5mu \mathbb{Z}\times[\mskip1.5mu \mathbb{Z}\mskip1.5mu]\mskip1.5mu]}
           \ar[ll]^-{\ensuremath{\mathbf{in}}}
           \ar[d]^-{\ensuremath{\Varid{id}\mathbin{+}(\Varid{id}\times\Varid{id})\times\llparenthesis\, \Varid{conquer}\,\rrparenthesis}}
\\
    \ensuremath{\mathbb{Z}\times[\mskip1.5mu \mathbb{Z}\mskip1.5mu]}
&
&
    \ensuremath{\mathrm{1}\mathbin{+}(\mathbb{Z}\times[\mskip1.5mu \mathbb{Z}\mskip1.5mu])\times(\mathbb{Z}\times[\mskip1.5mu \mathbb{Z}\mskip1.5mu])}
            \ar[ll]^-{\ensuremath{\alt{\conj{\underline{\mathrm{0}}}{\Varid{nil}}}{\Varid{cond}\;(\uncurry{(\mathrel{\wedge})}\comp \conj{\Varid{pp1}}{\Varid{pp2}})\;\p2\;\p1}}}
}
\end{eqnarray*}

\subsection*{Problema 2}
Primeira parte:

O \ensuremath{\Varid{outNat}} não é suficiente para este problema, logo desenvolvemos um \ensuremath{\Varid{outPrimes}} que junta os casos de 0, 1 e -1,
e no outro caso faz o módulo do número que recebe.

\ensuremath{\Varid{divisorsList}} cria a lista de divisores de um número até à sua raiz quadrada (otimização).
Esta é usada depois na função \ensuremath{\Varid{isPrime}} que verifica se um número é primo.
Depois, \ensuremath{\Varid{nextFactor}} verifica qual o próximo fator primo de um número.

Por fim, \ensuremath{\Varid{primes}} é o anamorfismo que gera a lista de primos. A parte \ensuremath{\Varid{aap}\;\cdot \div \cdot \;\Varid{nextFactor}} recebe um inteiro \ensuremath{\Varid{n}} e
divide-o pelo seu próximo fator primo.
\ensuremath{\Varid{aap}} é uma função monádica que calcula \ensuremath{\Varid{nextFactor}\;\Varid{n}} e depois usa o resultado como argumento a \ensuremath{\Varid{n}\div \Varid{nextFactor}\;\Varid{n}}.
\begin{hscode}\SaveRestoreHook
\column{B}{@{}>{\hspre}l<{\hspost}@{}}%
\column{E}{@{}>{\hspre}l<{\hspost}@{}}%
\>[B]{}\Varid{outPrimes}\;\mathrm{0}\mathrel{=}i_1\;(){}\<[E]%
\\
\>[B]{}\Varid{outPrimes}\;\mathrm{1}\mathrel{=}i_1\;(){}\<[E]%
\\
\>[B]{}\Varid{outPrimes}\;(\mathbin{-}\mathrm{1})\mathrel{=}i_1\;(){}\<[E]%
\\
\>[B]{}\Varid{outPrimes}\;\Varid{n}\mathrel{=}i_2\;(\Varid{abs}\;\Varid{n}){}\<[E]%
\ColumnHook
\end{hscode}\resethooks
\begin{hscode}\SaveRestoreHook
\column{B}{@{}>{\hspre}l<{\hspost}@{}}%
\column{5}{@{}>{\hspre}l<{\hspost}@{}}%
\column{E}{@{}>{\hspre}l<{\hspost}@{}}%
\>[B]{}\Varid{divisorsList}\mathbin{::}\mathbb{Z}\to [\mskip1.5mu \mathbb{Z}\mskip1.5mu]{}\<[E]%
\\
\>[B]{}\Varid{divisorsList}\;\Varid{n}\mathrel{=}[\mskip1.5mu \Varid{x}\mid \Varid{x}\leftarrow [\mskip1.5mu \mathrm{2}\mathinner{\ldotp\ldotp}\Varid{isqrt}\;\Varid{n}\mskip1.5mu],\Varid{mod}\;\Varid{n}\;\Varid{x}\equiv \mathrm{0}\mskip1.5mu]{}\<[E]%
\\
\>[B]{}\hsindent{5}{}\<[5]%
\>[5]{}\mathbf{where}\;\Varid{isqrt}\mathrel{=}\Varid{floor}\comp \Varid{sqrt}\comp \Varid{fromIntegral}{}\<[E]%
\ColumnHook
\end{hscode}\resethooks
\begin{hscode}\SaveRestoreHook
\column{B}{@{}>{\hspre}l<{\hspost}@{}}%
\column{E}{@{}>{\hspre}l<{\hspost}@{}}%
\>[B]{}\Varid{isPrime}\mathbin{::}\mathbb{Z}\to \Conid{Bool}{}\<[E]%
\\
\>[B]{}\Varid{isPrime}\mathrel{=}\alt{\Varid{false}}{\Varid{null}\comp \Varid{divisorsList}}\comp \Varid{outPrimes}{}\<[E]%
\ColumnHook
\end{hscode}\resethooks
\begin{hscode}\SaveRestoreHook
\column{B}{@{}>{\hspre}l<{\hspost}@{}}%
\column{E}{@{}>{\hspre}l<{\hspost}@{}}%
\>[B]{}\Varid{nextFactor}\mathbin{::}\mathbb{Z}\to \mathbb{Z}{}\<[E]%
\\
\>[B]{}\Varid{nextFactor}\;\Varid{n}\mathrel{=}\Varid{head}\;[\mskip1.5mu \Varid{x}\mid \Varid{x}\leftarrow \mathrm{2}\mathbin{:}[\mskip1.5mu \mathrm{3},\mathrm{5}\mathinner{\ldotp\ldotp}\Varid{n}\mskip1.5mu],\Varid{mod}\;\Varid{n}\;\Varid{x}\equiv \mathrm{0}\mathrel{\wedge}\Varid{isPrime}\;\Varid{x}\mskip1.5mu]{}\<[E]%
\ColumnHook
\end{hscode}\resethooks
\begin{hscode}\SaveRestoreHook
\column{B}{@{}>{\hspre}l<{\hspost}@{}}%
\column{E}{@{}>{\hspre}l<{\hspost}@{}}%
\>[B]{}\Varid{primes}\mathrel{=}\anaList{(\Varid{id}+\conj{\Varid{nextFactor}}{\Varid{aap}\;\cdot \div \cdot \;\Varid{nextFactor}})\comp \Varid{outPrimes}}{}\<[E]%
\ColumnHook
\end{hscode}\resethooks

\begin{eqnarray*}
\xymatrix@C=5cm@R=2cm{
    \ensuremath{\mathbb{Z}}
           \ar[d]_-{\ensuremath{\Varid{primes}}}
           \ar[r]^-{\ensuremath{\Varid{outPrimes}}}
&
    \ensuremath{\mathrm{1}\mathbin{+}\mathbb{Z}}
           \ar[r]^-{\ensuremath{\Varid{id}\mathbin{+}\conj{\Varid{nextFactor}}{\Varid{aap}\;\cdot \div \cdot \;\Varid{nextFactor}}}}
&
    \ensuremath{\mathrm{1}\mathbin{+}\mathbb{Z}\times\mathbb{Z}}
              \ar[d]^-{\ensuremath{\Varid{id}\mathbin{+}\Varid{id}\times\Varid{primes}}}
\\
     \ensuremath{[\mskip1.5mu \mathbb{Z}\mskip1.5mu]}
&
&
     \ensuremath{\mathrm{1}\mathbin{+}\mathbb{Z}\times[\mskip1.5mu \mathbb{Z}\mskip1.5mu]}
           \ar[ll]^-{\ensuremath{\alt{\Varid{nil}}{\Varid{cons}}}}
}
\end{eqnarray*}

Segunda parte:

Para criar a árvore precisamos criar os 'ramos', neste caso num par no formato (fatorização, número) com a função \ensuremath{\Varid{buildPairs}}.

Depois, fornecemos esses 'ramos' à função \ensuremath{\Varid{untar}} que irá criar a árvore do tipo \ensuremath{\Conid{Exp}\;\mathbb{Z}\;\mathbb{Z}}. Essa árvore vem numa lista, então apenas precisamos de usar \ensuremath{\Varid{head}} para a obter.
\begin{hscode}\SaveRestoreHook
\column{B}{@{}>{\hspre}l<{\hspost}@{}}%
\column{E}{@{}>{\hspre}l<{\hspost}@{}}%
\>[B]{}\Varid{buildPairs}\mathbin{::}[\mskip1.5mu \mathbb{Z}\mskip1.5mu]\to [\mskip1.5mu ([\mskip1.5mu \mathbb{Z}\mskip1.5mu],\mathbb{Z})\mskip1.5mu]{}\<[E]%
\\
\>[B]{}\Varid{buildPairs}\mathrel{=}\map \;\conj{(\mathbin{:})\;\mathrm{1}\comp \Varid{primes}}{\Varid{id}}{}\<[E]%
\\[\blanklineskip]%
\>[B]{}\Varid{prime\char95 tree}\mathrel{=}\Varid{head}\comp \Varid{untar}\comp \Varid{buildPairs}{}\<[E]%
\ColumnHook
\end{hscode}\resethooks

\subsection*{Problema 3}
O primeiro passo para a implementação da convolução de 2 listas é acrescentar (\ensuremath{\length \;\Varid{l1}\mathbin{-}\mathrm{1}}) zeros ao início da segunda lista
para que fique com tamanho igual a \ensuremath{\length \;\Varid{l1}\mathbin{+}\length \;l_2 \mathbin{-}\mathrm{1}}, que é o tamanho da lista resultante da convolução.
Depois calculamos as sublistas com \ensuremath{\Varid{sufixes}} para simular o deslizar de uma lista sobre a outra.
De seguida são aplicados 2 \ensuremath{\map }, um escrito na forma de catamorfismo outro em anamorfismo para ser possível criar um hilomorfismo.
Primeiro, multiplica-se os elementos correspondentes das sublistas e da primeira lista invertida e de seguida reduz-se cada lista à sua soma
tendo como resultado a lista correspondente à convolução.

\begin{hscode}\SaveRestoreHook
\column{B}{@{}>{\hspre}l<{\hspost}@{}}%
\column{3}{@{}>{\hspre}l<{\hspost}@{}}%
\column{9}{@{}>{\hspre}l<{\hspost}@{}}%
\column{E}{@{}>{\hspre}l<{\hspost}@{}}%
\>[B]{}\Varid{convolve}\mathbin{::}\Conid{Num}\;\Varid{a}\Rightarrow [\mskip1.5mu \Varid{a}\mskip1.5mu]\to [\mskip1.5mu \Varid{a}\mskip1.5mu]\to [\mskip1.5mu \Varid{a}\mskip1.5mu]{}\<[E]%
\\
\>[B]{}\Varid{convolve}\;\Varid{l1}\mathrel{=}\Varid{hyloList}\;\Varid{f}\;\Varid{g}\comp \Varid{suffixes}\comp \Varid{flip}\;\Varid{padZeros}\;(\length \;\Varid{l1}\mathbin{-}\mathrm{1}){}\<[E]%
\\
\>[B]{}\hsindent{3}{}\<[3]%
\>[3]{}\mathbf{where}\;\Varid{padZeros}\;\Varid{l}\mathrel{=}\cataNat{\alt{\underline{\Varid{l}}}{\mathrm{0}\mathbin{:}}}{}\<[E]%
\\
\>[3]{}\hsindent{6}{}\<[9]%
\>[9]{}\Varid{f}\mathrel{=}\alt{\Varid{nil}}{\Varid{cons}\comp (\Varid{sum}\times\Varid{id})}{}\<[E]%
\\
\>[3]{}\hsindent{6}{}\<[9]%
\>[9]{}\Varid{g}\mathrel{=}(\Varid{id}+(\Varid{zipWith}\;(\mathbin{*})\;(\Varid{reverse'}\;\Varid{l1})\times\Varid{id}))\comp \Varid{outList}{}\<[E]%
\ColumnHook
\end{hscode}\resethooks

Diagrama \ensuremath{\Varid{padZeros}}:
\begin{eqnarray*}
\xymatrix@C=2cm@R=2cm{
    \ensuremath{\N_0}
           \ar[d]_-{\ensuremath{\Varid{padZeros}\;\Varid{l}}}
           \ar[r]^-{\ensuremath{\Varid{outNat0}}}
&
    \ensuremath{\mathrm{1}\mathbin{+}\N_0}
              \ar[d]^-{\ensuremath{\Varid{id}\mathbin{+}(\Varid{padZeros}\;\Varid{l})}}
\\
     \ensuremath{[\mskip1.5mu \mathbb{Z}\mskip1.5mu]}
&
     \ensuremath{\mathrm{1}\mathbin{+}[\mskip1.5mu \mathbb{Z}\mskip1.5mu]}
           \ar[l]^-{\ensuremath{\alt{\underline{\Varid{l}}}{\mathrm{0}\mathbin{:}}}}
}
\end{eqnarray*}

Diagrama \ensuremath{\Varid{convolve}}:
\begin{eqnarray*}
\xymatrix@C=3.5cm@R=1.5cm{
    \ensuremath{[\mskip1.5mu \Conid{A}\mskip1.5mu]}
           \ar[r]^-{\ensuremath{\Varid{flip}\;\Varid{padZeros}\;(\length \;\Varid{l1}\mathbin{-}\mathrm{1})}}
           \ar[drr]_-{\ensuremath{\Varid{convolve}\;\Varid{l1}}}
&
    \ensuremath{[\mskip1.5mu \Conid{A}\mskip1.5mu]}
            \ar[r]^-{\ensuremath{\Varid{suffixes}}}
&
    \ensuremath{[\mskip1.5mu [\mskip1.5mu \Conid{A}\mskip1.5mu]\mskip1.5mu]}
            \ar[r]^-{\ensuremath{\Varid{g}}}
            \ar[d]^-{\ensuremath{\Varid{hyloList}\;\Varid{f}\;\Varid{g}}}
&
    \ensuremath{\mathrm{1}\mathbin{+}[\mskip1.5mu \Conid{A}\mskip1.5mu]\times[\mskip1.5mu [\mskip1.5mu \Conid{A}\mskip1.5mu]\mskip1.5mu]}
            \ar[d]^-{\ensuremath{\Varid{id}\mathbin{+}\Varid{id}\times(\Varid{hyloList}\;\Varid{f}\;\Varid{g})}}
\\
&
&
     \ensuremath{[\mskip1.5mu \Conid{A}\mskip1.5mu]}
&
     \ensuremath{\mathrm{1}\mathbin{+}[\mskip1.5mu \Conid{A}\mskip1.5mu]\times[\mskip1.5mu \Conid{A}\mskip1.5mu]}
           \ar[l]^-{\ensuremath{\Varid{f}}}
}
\end{eqnarray*}

\subsection*{Problema 4}
Definição do tipo:
\begin{hscode}\SaveRestoreHook
\column{B}{@{}>{\hspre}l<{\hspost}@{}}%
\column{E}{@{}>{\hspre}l<{\hspost}@{}}%
\>[B]{}\Varid{outExpr}\;(\Conid{V}\;\Varid{a})\mathrel{=}i_1\;\Varid{a}{}\<[E]%
\\
\>[B]{}\Varid{outExpr}\;(\Conid{N}\;\Varid{b})\mathrel{=}i_2\;(i_1\;\Varid{b}){}\<[E]%
\\
\>[B]{}\Varid{outExpr}\;(\Conid{T}\;\Varid{op}\;\Varid{l})\mathrel{=}i_2\;(i_2\;(\Varid{op},\Varid{l})){}\<[E]%
\\[\blanklineskip]%
\>[B]{}\Varid{recExpr}\;\Varid{f}\mathrel{=}\Varid{baseExpr}\;\Varid{id}\;\Varid{id}\;\Varid{f}{}\<[E]%
\ColumnHook
\end{hscode}\resethooks
\emph{Ana + cata + hylo}:
\begin{hscode}\SaveRestoreHook
\column{B}{@{}>{\hspre}l<{\hspost}@{}}%
\column{E}{@{}>{\hspre}l<{\hspost}@{}}%
\>[B]{}\Varid{cataExpr}\;\Varid{g}\mathrel{=}\Varid{g}\comp \Varid{recExpr}\;(\Varid{cataExpr}\;\Varid{g})\comp \Varid{outExpr}{}\<[E]%
\\[\blanklineskip]%
\>[B]{}\Varid{anaExpr}\;\Varid{g}\mathrel{=}\Varid{inExpr}\comp \Varid{recExpr}\;(\Varid{anaExpr}\;\Varid{g})\comp \Varid{g}{}\<[E]%
\\[\blanklineskip]%
\>[B]{}\Varid{hyloExpr}\;\Varid{h}\;\Varid{g}\mathrel{=}\Varid{cataExpr}\;\Varid{h}\comp \Varid{anaExpr}\;\Varid{g}{}\<[E]%
\ColumnHook
\end{hscode}\resethooks

\begin{hscode}\SaveRestoreHook
\column{B}{@{}>{\hspre}l<{\hspost}@{}}%
\column{5}{@{}>{\hspre}l<{\hspost}@{}}%
\column{E}{@{}>{\hspre}l<{\hspost}@{}}%
\>[B]{}\mathbf{instance}\;\Conid{Functor}\;(\Conid{Expr}\;\Varid{b})\;\mathbf{where}{}\<[E]%
\\
\>[B]{}\hsindent{5}{}\<[5]%
\>[5]{}\mathsf{fmap}\;\Varid{f}\mathrel{=}\Varid{cataExpr}\;(\Varid{inExpr}\comp \Varid{baseExpr}\;\Varid{f}\;\Varid{id}\;\Varid{id}){}\<[E]%
\\[\blanklineskip]%
\>[B]{}\mathbf{instance}\;\Conid{Applicative}\;(\Conid{Expr}\;\Varid{b})\;\mathbf{where}{}\<[E]%
\\
\>[B]{}\hsindent{5}{}\<[5]%
\>[5]{}\Varid{pure}\mathrel{=}\Varid{return}{}\<[E]%
\\
\>[B]{}\hsindent{5}{}\<[5]%
\>[5]{}(\mathbin{<*>})\mathrel{=}\Varid{aap}{}\<[E]%
\\[\blanklineskip]%
\>[B]{}\mathbf{instance}\;\Conid{Monad}\;(\Conid{Expr}\;\Varid{b})\;\mathbf{where}{}\<[E]%
\\
\>[B]{}\hsindent{5}{}\<[5]%
\>[5]{}\Varid{return}\mathrel{=}\Conid{V}{}\<[E]%
\\
\>[B]{}\hsindent{5}{}\<[5]%
\>[5]{}\Varid{e}\bind \Varid{f}\mathrel{=}\Varid{cataExpr}\;\alt{\Varid{f}}{\alt{\Conid{N}}{\uncurry{\Conid{T}}}}\;\Varid{e}{}\<[E]%
\ColumnHook
\end{hscode}\resethooks
\emph{Maps}:
\emph{Monad}:
Monad:
\emph{Let expressions}:
\begin{hscode}\SaveRestoreHook
\column{B}{@{}>{\hspre}l<{\hspost}@{}}%
\column{E}{@{}>{\hspre}l<{\hspost}@{}}%
\>[B]{}\Varid{let\char95 exp}\mathrel{=}\Varid{flip}\;(\bind ){}\<[E]%
\ColumnHook
\end{hscode}\resethooks

\ensuremath{\Varid{let\char95 exp}} pode ser explicada pelo seguinte diagrama (que também ajuda a perceber \ensuremath{(\bind )}):
\begin{eqnarray*}
\xymatrix@C=3.5cm@R=2cm{
    \ensuremath{\Conid{Expr}\;\Conid{C}\;\Conid{A}}
           \ar[d]_-{\ensuremath{\Varid{let\char95 exp}\;\Varid{f}}}
           \ar[r]^-{\ensuremath{\Varid{outExpr}}}
&
    \ensuremath{\Conid{A}\mathbin{+}(\Conid{C}\mathbin{+}\Conid{Op}\times(\Conid{Expr}\;\Conid{C}\;\Conid{A}))}
           \ar[d]^-{\ensuremath{\Varid{id}\mathbin{+}(\Varid{id}\mathbin{+}\Varid{id}\;\Varid{x}\;(\map \;(\Varid{let\char95 exp}\;\Varid{f})))}}
\\
     \ensuremath{\Conid{Expr}\;\Conid{C}\;\Conid{B}}
&
     \ensuremath{\Conid{A}\mathbin{+}(\Conid{C}\mathbin{+}\Conid{Op}\times(\Conid{Expr}\;\Conid{C}\;\Conid{B}))}
           \ar[l]^-{\ensuremath{\alt{\Varid{f}}{\alt{\Conid{N}}{\uncurry{\Conid{T}}}}}}
}
\end{eqnarray*}

Catamorfismo monádico:
\begin{hscode}\SaveRestoreHook
\column{B}{@{}>{\hspre}l<{\hspost}@{}}%
\column{5}{@{}>{\hspre}l<{\hspost}@{}}%
\column{E}{@{}>{\hspre}l<{\hspost}@{}}%
\>[B]{}\Varid{mcataExpr}\;\Varid{g}\mathrel{=}\Varid{g}\mathbin{.!}(\Varid{aux}\comp \Varid{recExpr}\;(\Varid{mcataExpr}\;\Varid{g})\comp \Varid{outExpr}){}\<[E]%
\\[\blanklineskip]%
\>[B]{}\Varid{aux}\mathbin{::}\Conid{Monad}\;\Varid{m}\Rightarrow \Varid{a}+(\Varid{b}+(\Conid{Op},[\mskip1.5mu \Varid{m}\;\Varid{c}\mskip1.5mu]))\to \Varid{m}\;(\Varid{a}+(\Varid{b}+(\Conid{Op},\Varid{m}\;[\mskip1.5mu \Varid{c}\mskip1.5mu]))){}\<[E]%
\\
\>[B]{}\Varid{aux}\mathrel{=}\alt{\Varid{return}\comp i_1}{\alt{\Varid{return}\comp i_2\comp i_1}{\Varid{ret}}}{}\<[E]%
\\
\>[B]{}\hsindent{5}{}\<[5]%
\>[5]{}\mathbf{where}\;\Varid{ret}\mathrel{=}\Varid{return}\comp i_2\comp i_2\comp \conj{\p1}{\Varid{sequence}\comp \p2}{}\<[E]%
\ColumnHook
\end{hscode}\resethooks
Avaliação de expressões:
\begin{hscode}\SaveRestoreHook
\column{B}{@{}>{\hspre}l<{\hspost}@{}}%
\column{5}{@{}>{\hspre}l<{\hspost}@{}}%
\column{29}{@{}>{\hspre}l<{\hspost}@{}}%
\column{E}{@{}>{\hspre}l<{\hspost}@{}}%
\>[B]{}\Varid{evaluate}\mathrel{=}\Varid{cataExpr}\;\alt{\Varid{nothing}}{\alt{\Conid{Just}}{\Varid{g}}}{}\<[E]%
\\
\>[B]{}\hsindent{5}{}\<[5]%
\>[5]{}\mathbf{where}\;\Varid{g}\;(\Varid{op},\Varid{vals})\mathrel{=}\mathbf{case}\;(\Varid{op},\Varid{sequence}\;\Varid{vals})\;\mathbf{of}{}\<[E]%
\\
\>[5]{}\hsindent{24}{}\<[29]%
\>[29]{}(\Conid{Add},\Conid{Just}\;[\mskip1.5mu \Varid{x},\Varid{y}\mskip1.5mu])\to \Conid{Just}\;(\Varid{x}\mathbin{+}\Varid{y}){}\<[E]%
\\
\>[5]{}\hsindent{24}{}\<[29]%
\>[29]{}(\Conid{Mul},\Conid{Just}\;[\mskip1.5mu \Varid{x},\Varid{y}\mskip1.5mu])\to \Conid{Just}\;(\Varid{x}\mathbin{*}\Varid{y}){}\<[E]%
\\
\>[5]{}\hsindent{24}{}\<[29]%
\>[29]{}(\Conid{ITE},\Conid{Just}\;[\mskip1.5mu \Varid{cond},\Varid{t},\Varid{e}\mskip1.5mu])\to \mathbf{if}\;\Varid{cond}\mathbin{>}\mathrm{0}\;\mathbf{then}\;\Conid{Just}\;\Varid{t}\;\mathbf{else}\;\Conid{Just}\;\Varid{e}{}\<[E]%
\\
\>[5]{}\hsindent{24}{}\<[29]%
\>[29]{}\anonymous \to \Conid{Nothing}{}\<[E]%
\ColumnHook
\end{hscode}\resethooks

%----------------- Índice remissivo (exige makeindex) -------------------------%

\printindex

%----------------- Bibliografia (exige bibtex) --------------------------------%

\bibliographystyle{plain}
\bibliography{cp2425t}

%----------------- Fim do documento -------------------------------------------%
\end{document}
